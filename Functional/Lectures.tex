%% LyX 2.1.1 created this file.  For more info, see http://www.lyx.org/.
%% Do not edit unless you really know what you are doing.
\documentclass[english]{article}
\usepackage[T2A,T1]{fontenc}
\usepackage[cp1251]{inputenc}
\usepackage{geometry}
\geometry{verbose,tmargin=2cm,bmargin=2cm,lmargin=1cm,rmargin=1cm}
\usepackage{amsthm}
\usepackage{amsmath}

\makeatletter

%%%%%%%%%%%%%%%%%%%%%%%%%%%%%% LyX specific LaTeX commands.
\DeclareRobustCommand{\cyrtext}{%
  \fontencoding{T2A}\selectfont\def\encodingdefault{T2A}}
\DeclareRobustCommand{\textcyr}[1]{\leavevmode{\cyrtext #1}}
\AtBeginDocument{\DeclareFontEncoding{T2A}{}{}}


%%%%%%%%%%%%%%%%%%%%%%%%%%%%%% Textclass specific LaTeX commands.
  \theoremstyle{definition}
  \newtheorem{xca}{\protect\exercisename}[section]
  \theoremstyle{definition}
  \newtheorem{example}{\protect\examplename}[section]
  \theoremstyle{plain}
  \newtheorem{cor}{\protect\corollaryname}[section]
\newenvironment{lyxcode}
{\par\begin{list}{}{
\setlength{\rightmargin}{\leftmargin}
\setlength{\listparindent}{0pt}% needed for AMS classes
\raggedright
\setlength{\itemsep}{0pt}
\setlength{\parsep}{0pt}
\normalfont\ttfamily}%
 \item[]}
{\end{list}}
  \theoremstyle{remark}
  \newtheorem{note}{\protect\notename}
  \theoremstyle{remark}
  \newtheorem{rem}{\protect\remarkname}[section]

%%%%%%%%%%%%%%%%%%%%%%%%%%%%%% User specified LaTeX commands.
\usepackage[all]{xy}
\usepackage{pgf,tikz}
\usepackage{mathrsfs}
\usetikzlibrary{arrows}

\makeatother

\usepackage{babel}
\usepackage{listings}
  \providecommand{\examplename}{Example}
  \providecommand{\exercisename}{Exercise}
  \providecommand{\notename}{Note}
  \providecommand{\remarkname}{Remark}
\providecommand{\corollaryname}{Corollary}
\renewcommand{\lstlistingname}{Listing}

\begin{document}

\title{Функциональное программирование}


\author{Akhtyamov Pavel}

\maketitle

\section{Деревья}


\subsection{Деревья общего вида}

Деревья используются в структурах данных (к примеру, для двоичного
поиска), в синтаксическом анализе, и у них различная структура.

Дерево - элемент типа $T$ + список деревьев типа $T$.

\begin{lstlisting}
let rec map f = function
| Leaf(x) -> Leaf(f x)
| Node(x, l) -> Node(f x, List.map (map f) l)
\end{lstlisting}


Fold (правая свертка, раскрывается снизу вверх)

\begin{lstlisting}
let rec fold f i = function
| Leaf(x) -> f i x
| Node(x, l) -> l |> List.fold (fold f) (f i x)
\end{lstlisting}

\begin{xca}
Отображение - отрубить листья у дерева.\end{xca}
\begin{example}
Абстрактное синтаксическое дерево

\begin{lstlisting}
type 't Expr = 
| Add of 't * Expr * 't Expr
| Sub of 't * Expr * 't Expr
| Mul of 't * Expr * 't Expr
| Div of 't * Expr * 't Expr
| Neg of 't * Expr
| Value of 't

let rec compute = function
| Add(x, y) -> compute x + compute y
| Sub(x, y) -> compute x - compute y
| Mul(x, y) -> compute x * compute y
| Div(x, y) -> compute x / compute y
| Neg(x) -> -compute x
| Value(x) -> x
\end{lstlisting}


Как построить по списку дерево? Для этого есть разбор выражения в
дерево. На вход получить поток, взять столько, сколько смогло сработать
(а затем вернуть оставшуюся часть). Возвращается дерево и хвост, который
не обработан.
\end{example}
\begin{lstlisting}
let mknode = function
| '+' -> Add
| '-' -> Sub
| '*' -> Mul
| '/' -> Div

let parse_infix expr = 
	let rec ff = function
	| [] -> failwith "Error"
	| h::t when isDigit -> parse
	| h::t when h= ' ' -> ff t
	| h::t when h='~' -> // negate
		...
	| h::t -> ... // binary
\end{lstlisting}


Дерево может порождаться функцией (в не чисто функциональных языках).


\subsection{Двоичные деревья}

Двоичное дерево представляется в виде узла и двух детей.

Есть несколько обходов двоичного дерева:
\begin{itemize}
\item Префиксный
\item Инфиксный
\item Постфиксный
\end{itemize}
Можно заменить энергичные вычисления на ленивые (см. код).

Двоичное дерво - все элементы меньше --- слева, больше --- справа

Хвостовая рекурсия для дерева: вроде бы невозможно (нет линейности).
Но используется подход продолжения.

\begin{lstlisting}
let plus1 x f = f(x+1)
let times2 x f = f (x*2)
plus1 1 (fun x -> times 2 x (printfn "%d")) // order way used
\end{lstlisting}


Иногда удобно использовать аккумулятор вкупе с продолжением (для левого
поддерева считаем, для правого --- запоминаем, что должны вычислить).
\begin{cor}
Таким образом, хотя и дереьев нет в библиотеке, они позволяют организовать
достаточно интересные и богатые структуры данных.
\end{cor}

\section{Погружение в функциональную составляющую}

Очередь на списке --- это плохо (необходим кольцевой список). Надо
придумывать реализацию (Chris Okazaki). Нам надо помещать быстро элемент
в очередь и изымать начало. Подойдет реализация на двух стеках. Перестройка
амортизационно идет за $O(1)$.

\begin{lstlisting}
let tail(L,R) = 
match L with
| [x] -> (List.rev R, [])
| h::t -> (t, R)
\end{lstlisting}


Рассмотрим ZipperList (двунаправленный список с выделенным текущим
элементом, реализация --- alt+Tab и машина Тьюринга).

$Zipper$ может рассматриваться не только на списках, но и на других
структурах, в частности, на дереве (написать get для последовательности
элементов). Далее хотим, чтобы элемент был доступен корню (идея ---
поддерево будет перекидываться в левый список), т.е идем налево ---
запоминаем правое поддерево и наоборот. Таким образом, аналог $Zipper$
--- устройство директорий в проводнике.

$Zipper$ подчиняются дифференцированию: если $T(A,R)=1+A*R*R$, то
тогда $Z=2*A*R$, где $2=Left|Right$, $A$ --- сохраняемый корневой
элемент, $B$ --- путь.


\subsection{Числовые структуры данных}

Сложение можно ассоциировать со списками и, к примеру, ординалами.
При этом сложение будет выполняться за $O(n)$. Но в десятичной системе
числа складывать удобнее. Числовые структуры данных напоминают числа
и группировку данных по образцу. Оптимальная позиционная система счисления
--- по основанию $e$. Но используют двоичные списки.

\begin{lstlisting}
type 'a blist = 
| Nil
| Zero of ('a*'a) blist
| One of 'a * ('a * 'a) blist
\end{lstlisting}


Nil --- 0
One (1, Nil) ---1
Zero(One (1,2), Nil) ---2, и т.д.


\subsection{Замыкания}

Рассмотрим функцию вычисления производной:

\begin{lstlisting}
let deriv f (dx: float) = 
	fun x -> (f(x + dx) - f(x))/dx
\end{lstlisting}


Но проблемы в том, что $f$ и $dx$ запомнились. В итоге, в функциональных
языках мы можем пользоваться функциями с внутренними состояниями.
А когда можно забыть переменную, которую мы захватили? Нужен $Garbage\ Collector$!
Но замыкания достаточно сложно реализуемы.

Как можно пользоваться замыканиями?

В $F\#$ есть механизм, позволяющий использовать динамическое связывание.

\begin{lstlisting}
let mutable x = 4
let adder y = x + y
adder 1
x <- 3
adder 1
\end{lstlisting}
Из примера видно, что сохраняется ссылка.

Замыкание возникает при частичном применении функции. При $(+)\ 1$
есть ссылка на библиотечную функцию.

Рассмотрим специализацию:

Программа: $P:I\rightarrow O$.

Interpreter: $Source\rightarrow Input\rightarrow Output$

Compiled=$(Interpreter\ Source):\ Input\rightarrow Output$. На таком
подходе $Futamura\ Projection$ возник подход суперкомпляции (лазить
внутри кода и оптимизация в нем).


\subsection{Генераторы}

Файл больше самого большого массива (ибо нам надо последовательный
участок памяти). Но ведь можно считать все посимвольно, и возникает
понятие генератора (производить последовательность вычислений, возможно,
бесконечную). Это можно реализовать на императивно-функциональном
языка (взято с $Lisp$).

\begin{lstlisting}
type cell = {mutable content: int}
let new_counter n =
	let x = {content = n} in
	fun () -> (x.content <- x.conten + 1; x.content)
\end{lstlisting}


Можно использовать $ref$. Счетчик --- частный случай генератора.

\begin{lstlisting}
let new_generator fgen init = 
	let x = ref init in
	fun () -> (x:= fgen |x; |x)

let fib = new_generator (fun(u,v) -> (u+v, u)) (1,1) // pairs

let map f g = fun() -> f(g())
let fibs = (fib |> map fst)
\end{lstlisting}


Можно описать функцию $filter$.

С помощью генераторов можно генерировать бесконечные последовательности.


\section{Монады}

Монады необходимы, к примеру, для организации ввода-вывода (технология
была создана в 1991 году).

К примеру, есть куча проверок на $null$. 
\begin{lyxcode}
let~read()~=~
\begin{lyxcode}
printf~``>''

let~s~=~Console.Readline()

try
\begin{lyxcode}
Some(int(s))
\end{lyxcode}
with
\begin{lyxcode}
\_~->~None
\end{lyxcode}
\end{lyxcode}
\end{lyxcode}
Для сложения двух чисел в таком коде необходимо выполнить много проверок
(если надо было бы складывать 10 чисел, то получилось плохо). Хотелось
бы ввести число, сложить его, и так далее (отделить смысловой алгоритм
от промежуточных проверок).
\begin{lyxcode}
let~bind~a~f~=~
\begin{lyxcode}
if~a~=~None~then~None

else~f~(a.Value)
\end{lyxcode}
bind~(read())~(fun~x~->~bind~(read())~(fun~y~->~Some(x~+~y))
\end{lyxcode}
Или
\begin{lyxcode}
let~(>\textcompwordmark{}>=)~a~f~=~
\begin{lyxcode}
if~a~=~None~then~None

else~f~(a.Value)
\end{lyxcode}
read()~>\textcompwordmark{}>=~fun~x~->~read()~>\textcompwordmark{}>=~fun~y~->~Some(x~+~y)
\end{lyxcode}
Код был приведен к простому виду.

Недетерминированные вычисления: Вася 2000 года рождения, Петя в два,
либо в 3 раза старшу, а Лена на год старше Пети. Сколько лет Лене?
\begin{lyxcode}
let~(>\textcompwordmark{}>=)~mA~f~=~List.collect~f~mA

{[}10;11{]}~>\textcompwordmark{}>=~(fun~x~->~{[}x{*}2;~x{*}3{]})~>\textcompwordmark{}>=~(fun~x~->~{[}x+1{]})
\end{lyxcode}
Как видно, что обработка скинута в bind.

Можно ввести операции типа return:
\begin{lyxcode}
let~ret~x~=~{[}x{]}

let~ret'~x~=~x

let~fail~=~{[}{]}

ret'~{[}10;11{]}~>\textcompwordmark{}>=~(fun~x~->~ret'~{[}x{*}2;x{*}3{]})~>\textcompwordmark{}>=~(fun~x~->~ret~(x~+~1))
\end{lyxcode}
Перейдем к определению монад:
\begin{itemize}
\item Для каждого базового типа есть обрамляющий тип M<t>
\item Есть операция return: 't -> M<'t>
\item Есть операция композиции bind:

\begin{itemize}
\item $>>=:M<'a>\rightarrow(a->M<'b>)\rightarrow M<'b>$
\end{itemize}
\end{itemize}
Таким образом, монады описывают последовательное применение вычислений,
но логика спрятана внутри (можно спрятать всю магию в ``;'')

Монада MayBe:
\begin{lyxcode}
//type~'t~option

return~x~=~Some(x)

bind~//аналогично~первому~примеру
\end{lyxcode}
Монады должна удовлетворять свойствам:
\begin{lyxcode}
return~x~>\textcompwordmark{}>=~f~$\equiv$~f~x

m~>\textcompwordmark{}>=~return~$\equiv$~m

(m~>\textcompwordmark{}>=~f)~>\textcompwordmark{}>=g~$\equiv$~m~>\textcompwordmark{}>=~($\lambda$x.fx~->~g(x))~//~ассоциативность~выполнения~операций
\end{lyxcode}
Как спрятать это дело от пользователя в F\#:
\begin{lyxcode}
let~r~=~nondet~\{
\begin{lyxcode}
let!~vasya~=~{[}14;15{]}

let!~petya~=~{[}2{*}vasya;~3{*}vasya{]}

let~lena~~=~petya~+~1

return~lena
\end{lyxcode}
\}
\end{lyxcode}
Весь bind запрятан в nondet (let' --- аналог ret')
\begin{lyxcode}
type~NonderBuilder()~=~
\begin{lyxcode}
number~b.Return(x)~=~ret~x

number~b.ReturnFrom(x)~=~ret'~x

number~b.Bind(mA,~f)~=~ma~>\textcompwordmark{}>=~f
\end{lyxcode}
let~nondet~=~new~NondetBuilder()
\end{lyxcode}
Помимо return и bind, можно описывать другие свойства (можно определить
удаление списка)
\begin{lyxcode}
let~rec~remove~x~l~=~
\begin{lyxcode}
nondet~\{
\begin{lyxcode}
if~l~=~{[}{]}~then~return~{[}{]}

else
\begin{lyxcode}
if~(List.head~l)~=~x~then~return~(List)...~//нет~else~->~реализуется~через~zero
\end{lyxcode}
\end{lyxcode}
\}
\end{lyxcode}
\end{lyxcode}
С помощью монад можно решать логические задачи (задача про льва и
единорога: можно перебрать все дни). 
\begin{itemize}
\item Надо получить тройку для вчерашнего дня
\item Проверить выражение на правдивость
\item Совершить решение:\end{itemize}
\begin{lyxcode}
nondet~\{
\begin{lyxcode}
let~(l,e,d)~=~data

let~(l1,~e1,~d1)~=~prev~(true,~true,~``sun'')~(fun(\_,\_,x)~->~x~=~d)~data

if~(realday~l~false)~=~l1~\&\&~(realday~e~false)~=~e1~then~return~(l,e,d)
\end{lyxcode}
\}~
\end{lyxcode}
По сути, мы решили задачу, которую можно решать в логической парадиме.

Рассмотрим примеры классических монад:

а) Монады ввода-вывода --- состояние ввода и вывода есть пара списков
\begin{lyxcode}
type~State~=~string~list~{*}~string~list
\begin{lyxcode}
type~IO~<'t>~=~'t{*}State

let~print(t:string)~((I,O):State)~=~(t,~(I,~t::O))

read~((I,O):state)~//~передвижение~каретки
\end{lyxcode}
\end{lyxcode}
б) Монады состояния: пример --- проход по дереву (тупая реализация
--- протаскивать глобальную переменную). С помощью state можно протащить
все данные внутри:
\begin{lyxcode}
State<'t>~=~'s~->~'t~{*}~'s

return~x~=~fun~s~->~x,~s

let~(>\textcompwordmark{}>=)~x~f~=~(fun~s0~->~let~a,~s~=~x~s0;~f~a~s)

let~getState~=~(fun~s~->~s,~s)

let~setState~s~=~(fun~\_~->~(),~s)

//do!~-~отбросить~unit~-~setState~пишется~с~do
\end{lyxcode}
в) Вычисления с продолжениями (скинуть хвостовую рекурсию в дереве
или определить порядок вычислений в ленивом языке) --- можно сказать,
что не надо возвращаться
\begin{lyxcode}
let~(>\textcompwordmark{}>=)~x~f~=~x~f

read~>\textcompwordmark{}>=~double~>\textcompwordmark{}>=~print~>\textcompwordmark{}>=~(fun~x~->~x)
\end{lyxcode}
Быстрая сортировка с хвостовой рекурсией:
\begin{lyxcode}
let~qsort~list~=~
\begin{lyxcode}
let~rec~sort~list~cont~=~
\begin{lyxcode}
match~list~with

|~{[}{]}~->~cont~{[}{]}

|~x::xs~->~
\begin{lyxcode}
let~l,r~=~
\begin{lyxcode}
List.partition~((>)~x~)~xs
\end{lyxcode}
sort~l~(fun~ls~->~sort~r~(fun~rs~->~cont(ls~@~(x::rs))))
\end{lyxcode}
\end{lyxcode}
sort~list~(fun~x~->~x)
\end{lyxcode}
\end{lyxcode}
Но выглядит это не очень красиво. Надо прятать в монаду продолжения:
\begin{lyxcode}
let~qsort~list~=~
\begin{lyxcode}
let~rec~sort~list~=
\begin{lyxcode}
cont~\{
\begin{lyxcode}
match~list~with

|~{[}{]}~->~return~{[}{]}

|~x::xs~->~
\begin{lyxcode}
let~l,r~=~
\begin{lyxcode}
List.partition~((>)~x~)~xs
\end{lyxcode}
let!~ls~=~sort~l

let!~rs~=~sort~r

return~ls~@~(x::rs)
\end{lyxcode}
\end{lyxcode}
\}
\end{lyxcode}
sort~list~(fun~x~->~x)
\end{lyxcode}
\end{lyxcode}
По сути, нехвостовая рекурсия заменена по внешнему виду на хвостовую,
но с протаскиванием монады.

\textbf{Рассмотрим монадические парсеры.}

Как решаются задачи разбора текста:
\begin{itemize}
\item По тексту строится синтаксическое дерево
\item Правила задаются с помощью грамматики
\item Возможные подходы:

\begin{itemize}
\item Лексический анализатор -> синтаксический анализатор
\item Руками
\item Монадический парсер (складываются из кусков - fparsec)
\end{itemize}
\end{itemize}
\begin{lyxcode}
let~pair~=~parse~\{
\begin{lyxcode}
let!~\_~=~str\_ws~``(``

let!~n1~=~int\_ws

let!~\_~=~str\_ws~``,''

...
\end{lyxcode}
\}

let~str\_ws~str~=~parse~\{
\begin{lyxcode}
do!~skipString~str

do!~spaces

return~()
\end{lyxcode}
\}

let~int\_ws~=~parse~\{
\begin{lyxcode}
...
\end{lyxcode}
\}
\end{lyxcode}

\subsection{Параллельные и асинхронные вычисления}

В функциональных языках можно вычислять что-то параллельно удобным
образом. Но, в целом, средства из коробки реализовать сложно (в Haskell,
к примеру).

Можно параллелить вручную: ставить Semaphore, Locks, etc. Есть расширения
из .NET.

Основные проблемы асинхронных и параллельных вычислений:
\begin{itemize}
\item Общая память (решается легко в функциональных языках)
\item Инверсия управления (надо вызывать функцию в функцией и т.д.)
\end{itemize}
В F\# есть Asynchronous Workflows:
\begin{lyxcode}
let~task1~=~async~\{return~10~+~10\}

let~task2~=~async~\{return~20~+~20\}

Async.RunSynchronously~\{Asyc.Parallel~{[}task1;~task2{]}\}

let~map'~func~items~=
\begin{lyxcode}
let~tasks~=~
\begin{lyxcode}
seq~\{
\begin{lyxcode}
for~i~in~items~->~async~\{~return~(func~i)~\}
\end{lyxcode}
\}
\end{lyxcode}
Async.RunSynchronously~(Async.Parallel~tasks)
\end{lyxcode}
\end{lyxcode}
В обычных языках код плохой: создавать каждый раз поток. Но в F\#
есть планировщик потоков, который не создает новые потоки, а раздает
потокам задания.

Асинхронность вкладывается в данный контекст:
\begin{lyxcode}
let~httpAsync(url:string)~=

async~\{
\begin{lyxcode}
let~req~=~WebRequest.Create(url)

let!~resp~=~req.AsyncGetResponse()

use~stream~=~resp.GetResponseStream()

use~reader~=~new~StreamReader(stream)

let!~text~=~Async.SwaitTask(reader.ReadToEndAsync())

return~text
\end{lyxcode}
\}
\end{lyxcode}
Async имеет монадический синтаксис (хороший синтаксис продолжения).
Реализация будет короче и проще, чем в C\#. Таким образом, решена
проблема инверсии.

Асинхронное чтение с диска:
\begin{lyxcode}
let~Readfile~fn~

let~goodfiles

let~goodsplit

let~wc~s~=~s~|>~goodsplit~|>~Seq.filter(fun~s~->~s.Length~>~0)~|>~Seq.length~

goodfiles~|>~Seq.map~(fun~f~->~(f,~wc~f))~|>~Seq.toArray

let~ReadfileAsync~fn~//заменить~read~на~asyncRead

goodfiles~
\begin{lyxcode}
|>~Seq.map~(fun~f~->~async~\{
\begin{lyxcode}
let!~s~=~ReadFileAsync~f

return~(f,~wc~s)

\})
\end{lyxcode}
|>~Async.Parallel

|>~Async.RunSyncronously
\end{lyxcode}
\end{lyxcode}
Программа проработала очень быстро.

Проблема: задачи распараллеливаются, если уже есть массив данных.
Но в реальной жизни бывает не так: есть сложная система, возникает
проблема масштабирования. Можно использовать паттерн Agent.
\begin{lyxcode}
let~rec~agent~=~MailboxProcessor.Start(fun~inbox~->
\begin{lyxcode}
let~rec~loop()~=~async~\{
\begin{lyxcode}
let!~msg~=~...

return!~loop()
\end{lyxcode}
\}

loop()
\end{lyxcode}
)
\end{lyxcode}
Как написать Crawl?
\begin{lyxcode}
let~GetLinks~s~=~

let~Crawl~url~=~
\begin{lyxcode}
async~\{
\begin{lyxcode}
let!~s~=~httpAsync~url

return~GetLinks~s
\end{lyxcode}
\}
\end{lyxcode}
let~CrawlAgent~=~MailboxProcessor.Start(fun~inbox~->
\begin{lyxcode}
let~rec~loop()~=~sync~\{
\begin{lyxcode}
let!~msg~=~inbox.Receive()

let!~res~=~Crawl~msg

res~|>~Seq.iter~(fun~x~->~CrawlerAgent.Post~x)

return!~loop()
\end{lyxcode}
\}

loop()
\end{lyxcode}
)

let~Dispatcher~n~f~=~//~параллельный~планировщик~агентов,~который~обрабатывает~пул~сообщений
\end{lyxcode}
Таким образом, монады дают прекрасный способ описывать параллельные
вычисления.
\begin{xca}
Написать частотный словарь Рунета (можно красиво написать на основе
реактивного программирования).
\end{xca}
Иногда можно вычислять параллельные задачи на облаке (кластерах).
Есть проект \{m\}brace, который помогает рассмотреть данный вопрос.
Для этого есть монада cloud. На кластере есть специальный подход к
вычислениям, который воспринимается им при помощи программы. (m-brace.net,
briskengine.com). Такой кластер можно развернуть на Microsoft Azure.

Можно это сделать другим образом (при помощи CloudFlow). Здесь использована
идея Map-Reduce, которая хорошо используется на облачных серверах. 

Внутренние переменные внутри cloud автоматически загружаются в облако
(чеерз cloudFlow можно загрузить все вручную).


\section{Реактивное функциональное программирование}

В прошлом разделе мы написали Crawl, который индексирует веб-сайты.
В нем использовался паттерн Agent. Как измерить доброту Интернета?
Не очень красивое решение: можно запилить функцию (но функцию надо
будет менять). 

Есть некоторые механизмы обработки:
\begin{itemize}
\item списки; (обработка в памти)
\item последовательности; (применять вытаскивание следующего элемента)
\item потоки; (вызов функциии на каждом элементе)\end{itemize}
\begin{lyxcode}
type~Stream<'T>~=~('T~->~unit)~->~unit
\end{lyxcode}
По сути, данные передаются в поток, а затем применяются функции (причем
про input мы не знаем информации).

Поэтому хочется, чтобы Crawler возвращал поток.

Реактивное программирование --- парадигма, ориентированнная на потоки
данных и распространение изменений, типичный пример --- таблицы в
Excel. Reactive Manifesto --- реактивный стиль в классических языках
программирования.

Реактивное функциональное программирование бывает дискретным и непрерывным.
\begin{lyxcode}
type~Crawler()~=~
\begin{lyxcode}
let~evt~=~new~Event<\_>()

...
\end{lyxcode}


evt.Trigger~-{}-{}-~отправляет~события

evt.Publish~-{}-{}-~подписка~(несколько~получателей~на~одно~событие).
\end{lyxcode}
Доброта Твиттера: создаем event, далее создаем listener, который получает
данные в виде JSON (работа с которым хороша с помощью NewtonSoft.JSON).
В данном случае в FSharpChart дается на вход Event, и она отображается
в виде FSharpChart.FastLine

Еще одним примером является естественный контроль движений руки (при
помощи Kinect). На каждом шаге передаем фрейм с устройства и контроллер,
можно посчитать скорость пальцев или тренажер пальцев при помощи функции
Event.Scan (аккумулировать максимум). Еще можно посмотреть при помощи
контроллера на 5 фреймов назад и проверять движения руки на жесты.

При помощи Kinect можно получать результаты разных упражнений.

Еще можно использовать реактивное программирование на облаке (каждый
кластер обрабатывает данные, а затем их отдает на узел, который складывается
в очередь и обрабатывается).

Event --- самый простой вид обработки, но что делать, если возникают
ошибки? В .NET есть Reactive Extensions: 
\begin{lyxcode}
IOObserver<T>(OnNext,~OnError),~IOObservable<T>
\end{lyxcode}
Как можно программировать робота? Сенсоры предоставляют Observable
объекты:
\begin{lyxcode}
//init~models

let~power~=~~sensor.ToObservable().Select(dist).DistinctUntilChanged()

use~l~=~power.Subscribe(motorL)

use~r~=~power.Subscribe(motorR)
\end{lyxcode}
На робот можно посылать данные с мобильных устройств (логика на клиенте,
на роботе --- простая работа): подписываем первую координату на один
мотор, вторую --- на другой мотор. Создается TrikPad, который позволяет
по наклону руки получает данные, которые надо послать на робота.

В итоге, реактивное программирование является хорошим способом создания
надежных систем регаирования в реальном времени. Еще реактивный подход
позволяет работать с потоками событий с использованием привычных примитивов.

Хорошим примером является обработка событий с мышки в пользовательском
интерфейсе, а еще есть хорошее применение в real-time анализе котировок.


\section{Реализация функциональных языков программрования}

Рассмотрим возможные реализации функциональных языков программирования.

Функциональный язык представляет расширенный функционал относительно
$\lambda$-исчисления. А здесь есть различия:
\begin{itemize}
\item Прямая интерпретация (хороша для энергичных языков и если не гонимся
за производительностью языков)
\item Редукция графов (изначально есть дерево, а далее идет редукция до
необходимого состояния, иногда доходит до графов)
\item Использование комбинаторов --- комбинаторная редукция
\item Код абстрактной машины --- на их основе можно построить интерпретатор
абстрактной машины (примеры SECD, .NET и т.п.)\end{itemize}
\begin{lyxcode}
type~Expr~=~
\begin{lyxcode}
|~App~of~Expr{*}Expr

|~Lam~of~string{*}Expr

|~Var~of~string

|~Cons~of~string

|~Clo~of~string{*}Expr{*}Env~//лямбда~с~захваченным~окружением
\end{lyxcode}
and~Env~=~Map<string,~Expr>

let~text~=~App(App(Lam(``f'',~Lam(``x'',~App(var...)))))

let~rec~eval~env~expr~=~

match~expr~with~
\begin{lyxcode}
|~App(e1,~e2)~->
\begin{lyxcode}
let~f1~=~eval~env~e1

let~f2~=~eval~env~e2

match~f1~with
\begin{lyxcode}
|~Clo(x,~ex,~env')~->~eval(Map.add~x~f2~env')~ex~//~добавление~в~окружение

|~Cons(x)~->~App(Cons(x),~f2)
\end{lyxcode}
\end{lyxcode}
|~Var(x)~->~Map.find~x~env

|~Cons(x)~->~Cons(x)

|~Lam(x,~e)~->~Clo(x,~e,~env)
\end{lyxcode}
eval~Map.empty~text
\end{lyxcode}
В Lisp интерпретатор состоит из eval и apply:
\begin{lyxcode}
|~Let~of~id{*}expr{*}expr

|~Pfunc~of~id~//~примитивная~функция

|~Op~of~id{*}int{*}expr~list~//~отложенные~операции,~int~-{}-{}-~число~нехватаемых~выражений
\end{lyxcode}
Eval:
\begin{lyxcode}
let~rec~eval~exp~env~=~
\begin{lyxcode}
match~exp~with

|~App(e1,~e2)~->~apply~(eval~e1~env)~(eval~e2~env)

|~Int(n)~->~Int(n)

|~Var(x)~->~Map.find~x~env

|~PGunc(f)~->~Op(f,~arity~f,~{[}{]})

...
\end{lyxcode}
let~apply~e1~e2~=~
\begin{lyxcode}
match~e1~with
\begin{lyxcode}
|~Closure(Lam(v,~e),~env)~->~eval~e~(Map.add~v~e2~env)

|~Op(id,~n,~args)~->
\begin{lyxcode}
if~n~=~1~then~(funof~id)(args@{[}e2{]})

else~Op(id,~n01,~args@{[}e2{]})
\end{lyxcode}
\end{lyxcode}
\end{lyxcode}
\end{lyxcode}
Таким образом был устроен интерпретатор Lisp.

Возникает вопрос: как работать с рекурсией?
\begin{lyxcode}
letrec~fact~=~

fun~x~->~
\begin{lyxcode}
if~x~<=~1~then~1

else~x{*}fact(x-1)
\end{lyxcode}
in~fact~4
\end{lyxcode}
Есть другой комбинатор неподвижной точки для энергичных вычислений,
но он огромный. Нам надо, чтобы в правой части вместо fact подставилось
определение этой функции. Надо придумать let, который подставляет
окружение функции (нужно добавить рекурсивное замыкание).
\begin{lyxcode}
eval~...

|~LetRec(id,~e1,~e2)~->~eval~e2~(Map.add~id~(RClosure(e1,~env,~id))~env)

apply~...

|~RClosure(Lam(v,~e),~env,~id)~->~eval~e~(Map.add~v~e2~(Map.add~id~e1,~env))~//~закрутка~выражений?
\end{lyxcode}
Как оказывается, letrec не тоже самое, что и let. 

Как добавить ленивые вычисления? Сначала вычисляем $f$, а вычисление
$x$ откладываем. Для откладывания вводится операция Susp:
\begin{lyxcode}
|App(e1,~e2)~->~apply(eval~e1~env)~(Susp(e2,~env))

...~some~changes~with~primitive~functions
\end{lyxcode}
Перейдем к рассмотрению реализации при помощи абстрактных машин. Прямая
интепретация дерева не очень эффективна (долгий обход по дереву).
Поэтому надо эффективно обходить дерево. Классическим примером является
язык SECD:
\begin{itemize}
\item S - Stack -- стек объектов для вычисления выражений
\item E - Environment -- среда для означивания переменных, контекст
\item C - Сontrol -- управляющая строка, оставшаяся часть вычисляемого выражения
\item D - Dump -- стек возвратов для обработки вложенных контекстов (вызовы
функций).
\end{itemize}
Все абстрактные машины являются стековыми, о количестве аргументов
заботиться не надо на этапе свертке (очень схоже с алгоритмом прохода-свертки
и LR-грамматики).

Данные операции выполняются аналогично eval-apply подходу, но оперируем
более простыми выражениями и идем итерационными методами.

Аналогично предыдущим реализациям можно рассмотреть остальные случаи
(рекурсия, условия и т.п.).

Рассмотрим редукцию графов. Как устроены правила редукции: ищется
самый низкий уровень лямбды, далее смотрим на переменную, а затем
меняем ссылки в правой части.
\begin{note}
The implementation of Functional Programming Languages, 1987
\end{note}
Реализация на потоковых графах: создается взаимодействие функций в
виде комбинируемых потоков из схемы функциональных элементов.

Мы не рассмотрели комбинаторную редукцию графов и реализацию на комбинаторной
логике.

Как можно реализовать грамматику: происходит лексический анализ, а
далее по ним строится синтаксический анализ (к примеру, методом рекурсивного
спуска). Но так делать плохо.

Поэтому есть fslex и fsyacc -- инструменты для построения всего этого
дела, аналогично yacc и lex. В lex строятся токены, а в yacc строится
грамматика.

Такой подход не очень удобен (ибо код генерируется плохой).
\begin{rem}
Разве? (Зам. автора)

В F\# есть комбинаторный парсер FParsec\end{rem}
\begin{lyxcode}
let~str~s~=~pstring~s

let~float\_in\_brackets~=~str~``(``~>\textcompwordmark{}>.~pfloat~.>\textcompwordmark{}>~str~``)''~//~>\textcompwordmark{}>.~-{}-~return~right,~.>\textcompwordmark{}>~-{}-~return~left,~.>\textcompwordmark{}>.~-{}-~pair

let~floats~=~sepBy~pfloat(str~``,'')

...

let~pexpr~,~pexpr\_ref~=~createParserForwarderToRef()

let~str'~s~=~str~s~.>\textcompwordmark{}>~ws

let~ident~=~indetifier~(new~IdentifierOptions())~.>\textcompwordmark{}>~ws

let~pid~=~ident~|>\textcompwordmark{}>~Var

let~plet~=~pipe~(str'~``let'')~ident~(str~''='')...

...

do~pexpr\_ref~:=~choice~{[}pid;~plet;~pint32~|>\textcompwordmark{}>~Int;~plam;~papp;~pfunc{]}

\end{lyxcode}

\end{document}
